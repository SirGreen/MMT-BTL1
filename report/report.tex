\documentclass[a4paper]{article}
\usepackage{vntex}
%\usepackage[english,vietnam]{babel}
%\usepackage[utf8]{inputenc}

%\usepackage[utf8]{inputenc}
%\usepackage[francais]{babel}


\usepackage{float}
\usepackage{a4wide,amssymb,epsfig,latexsym,multicol,array,hhline,fancyhdr}
\usepackage{booktabs}
\usepackage{amsmath}
\usepackage{lastpage}
\usepackage[lined,boxed,commentsnumbered]{algorithm2e}
\usepackage{enumerate}
\usepackage{color}
\usepackage{graphicx}							% Standard graphics package
\usepackage{array}
\usepackage{tabularx, caption}
\usepackage{multirow}
\usepackage[framemethod=tikz]{mdframed}% For highlighting paragraph backgrounds
\usepackage{multicol}
\usepackage{rotating}
\usepackage{graphics}
\usepackage{geometry}
\usepackage{setspace}
\usepackage{epsfig}
\usepackage{tikz}
\usepackage{listings}
\usetikzlibrary{arrows,snakes,backgrounds}
\usepackage{hyperref}
\usepackage{scrextend}
\changefontsizes{13pt}
\usepackage{svg}
\usepackage{hyperref}
\usepackage{enumitem}
\usepackage{tocloft}
\renewcommand{\cftsecleader}{\cftdotfill{\cftdotsep}}
\renewcommand{\thesection}{\Roman{section}}
\hypersetup{urlcolor=blue,linkcolor=black,citecolor=black,colorlinks=true} 
%\usepackage{pstcol} 								% PSTricks with the standard color package

\newtheorem{theorem}{{\bf Định lý}}
\newtheorem{property}{{\bf Tính chất}}
\newtheorem{proposition}{{\bf Mệnh đề}}
\newtheorem{corollary}[proposition]{{\bf Hệ quả}}
\newtheorem{lemma}[proposition]{{\bf Bổ đề}}

\everymath{\color{blue}}
%\usepackage{fancyhdr}
\setlength{\headheight}{40pt}
\pagestyle{fancy}
\fancyhead{} % clear all header fields
\fancyhead[L]{
 \begin{tabular}{rl}
    \begin{picture}(25,15)(0,0)
    \put(0,-8){\includegraphics[width=8mm, height=8mm]{hcmut.png}}
    %\put(0,-8){\epsfig{width=10mm,figure=hcmut.eps}}
   \end{picture}&
	%\includegraphics[width=8mm, height=8mm]{hcmut.png} & %
	\begin{tabular}{l}
		\textbf{\bf \ttfamily Trường Đại Học Bách Khoa Tp.Hồ Chí Minh}\\
		\textbf{\bf \ttfamily Khoa Khoa Học và Kỹ Thuật Máy Tính}
	\end{tabular} 	
 \end{tabular}
}
\fancyhead[R]{
	\begin{tabular}{l}
		\tiny \bf \\
		\tiny \bf 
	\end{tabular}  }
\fancyfoot{} % clear all footer fields
\fancyfoot[L]{\scriptsize \ttfamily Bài tập lớn môn Mạng máy tính - Niên khóa 2024-2025}
\fancyfoot[R]{\scriptsize \ttfamily Trang {\thepage}/\pageref{LastPage}}
\renewcommand{\headrulewidth}{0.3pt}
\renewcommand{\footrulewidth}{0.3pt}


%%%
\setcounter{secnumdepth}{4}
\setcounter{tocdepth}{3}
\makeatletter
\newcounter {subsubsubsection}[subsubsection]
\renewcommand\thesubsubsubsection{\thesubsubsection .\@alph\c@subsubsubsection}
\newcommand\subsubsubsection{\@startsection{subsubsubsection}{4}{\z@}%
                                     {-3.25ex\@plus -1ex \@minus -.2ex}%
                                     {1.5ex \@plus .2ex}%
                                     {\normalfont\normalsize\bfseries}}
\newcommand*\l@subsubsubsection{\@dottedtocline{3}{10.0em}{4.1em}}
\newcommand*{\subsubsubsectionmark}[1]{}
\makeatother

\definecolor{dkgreen}{rgb}{0,0.6,0}
\definecolor{gray}{rgb}{0.5,0.5,0.5}
\definecolor{mauve}{rgb}{0.58,0,0.82}
\definecolor{backcolor}{RGB}{212,235,242}
\lstset{frame=tb,
	language=R,
	aboveskip=3mm,
	belowskip=3mm,
	showstringspaces=false,
	columns=flexible,
	basicstyle={\small\ttfamily},
	numbers=none,
	numberstyle=\tiny\color{gray},
	keywordstyle=\color{blue},
	commentstyle=\color{dkgreen},
	stringstyle=\color{mauve},
	breaklines=true,
	breakatwhitespace=true,
	tabsize=3,
	numbers=left,
	stepnumber=1,
	numbersep=1pt,    
	firstnumber=1,
	numberfirstline=true,
}

\usepackage{indentfirst}
\begin{document}
\setlength{\parindent}{0.5cm}

\begin{titlepage}
\begin{center}
ĐẠI HỌC QUỐC GIA THÀNH PHỐ HỒ CHÍ MINH \\
TRƯỜNG ĐẠI HỌC BÁCH KHOA \\
KHOA KHOA HỌC - KỸ THUẬT MÁY TÍNH 
\end{center}

\begin{figure}[h!]
\begin{center}
\includegraphics[scale = 0.27]{01_logobachkhoasang.png}
\end{center}
\end{figure}


\begin{center}
\begin{tabular}{c}
	\multicolumn{1}{l}{\textbf{{\Large HỌC KỲ }}}\\
	~~\\
	\hline
	\\
	\multicolumn{1}{l}{\textbf{{\Large Bài tập lớn}}}\\
	\\
	
	\textbf{{\Huge Computer Network}}\\
	\\
	\hline
\end{tabular}
\end{center}

\begin{table}[h]
\begin{tabular}{rrll}
\hspace{2.5 cm} & GVHD: &  Hoàng Lê Hải Thanh\\
& SV: &  &\\
& &   Huỳnh Minh Khoa&- 2252346\\
& &Thân Nguyễn Minh Khoa &- 2252361\\
& & Trần Lương Yến Nhi &- 2252586\\
 & & Nguyễn Lê Vân Tú&- 2252881\\
\end{tabular}
\end{table}

\begin{center}
{\footnotesize TP. HỒ CHÍ MINH, THÁNG 10/2024}
\end{center}
\end{titlepage}

\newpage
\tableofcontents
\newpage

\section{BitTorrent File-Sharing Application Functions and Communication Protocols}

\begin{enumerate}[label=\textbf{\arabic*.}, leftmargin=30pt, itemsep=10pt]
    \item \textbf{File Discovery and Metadata Sharing}  
    \textit{Description:} Allows users to locate and retrieve metadata about files to be shared or downloaded, typically through .torrent files or magnet links.  
    \textit{Communication Protocols:}
    \begin{itemize}
        \item \textbf{HTTP:} For downloading .torrent files or accessing magnet links.
    \end{itemize}

    \item \textbf{Peer Discovery}  
    \textit{Description:} Enables the identification and connection to other peers sharing the same file, ensuring distributed sharing.  
    \textit{Communication Protocols:}
    \begin{itemize}
        \item \textbf{BitTorrent Tracker Protocol (HTTP):} Centralized approach to finding peers.
    \end{itemize}

    \item \textbf{File Piece Distribution}  
    \textit{Description:} Divides files into smaller pieces for efficient sharing. Each piece is individually shared and verified to ensure data integrity.  
    \textit{Communication Protocols:}
    \begin{itemize}
        \item \textbf{BitTorrent Protocol over TCP/UDP:} Manages the exchange of file pieces.
    \end{itemize}

    \item \textbf{Upload}  
    \textit{Description:} Handles the sending of file pieces from a user's device to multiple peers in the network. Ensures balanced contribution and resource sharing.  
    \textit{Communication Protocols:}
    \begin{itemize}
        \item \textbf{BitTorrent Protocol over TCP/UDP:} Governs the sending of data pieces to peers.
    \end{itemize}

    \item \textbf{Download}  
    \textit{Description:} Manages receiving file pieces from other peers. Ensures optimal use of network resources by downloading from multiple peers concurrently.  
    \textit{Communication Protocols:}
    \begin{itemize}
        \item \textbf{BitTorrent Protocol over TCP/UDP:} Controls receiving data pieces from multiple peers simultaneously.
    \end{itemize}

    \item \textbf{Piece Verification}  
    \textit{Description:} Verifies the integrity of each downloaded piece by comparing its hash to the expected value, ensuring the reliability of the download.  
    \textit{Communication Protocols:}
    \begin{itemize}
        \item \textbf{Internal Hash Verification (SHA-1):} Conducted within the application to confirm the accuracy of received data.
    \end{itemize}

    \item \textbf{Tit-for-Tat Function}  
    \textit{Description:} Implements a strategy to ensure fair sharing by prioritizing peers that contribute more data. Encourages reciprocation among peers to balance the network load.  
    \textit{Communication Protocols:}
    \begin{itemize}
        \item \textbf{BitTorrent Protocol:} Uses a built-in mechanism to manage upload/download ratios and prioritize reciprocating peers.
    \end{itemize}

    \item \textbf{Encryption Function}  
    \textit{Description:} Ensures secure data transfer by encrypting communications between peers. Uses key exchange mechanisms to establish a secure connection.  
    \textit{Communication Protocols:}
    \begin{itemize}
        \item \textbf{Diffie-Hellman Key Exchange:} Establishes a shared secret between peers to encrypt communication.
        \item \textbf{AES (Advanced Encryption Standard):} Encrypts the actual data transfer for security.
    \end{itemize}

    \item \textbf{Error Handling and Recovery}  
    \textit{Description:} Detects and recovers from issues like incomplete or corrupted downloads. Retries failed downloads and re-requests missing pieces from other peers.  
    \textit{Communication Protocols:}
    \begin{itemize}
        \item \textbf{BitTorrent Protocol:} Contains mechanisms for retrying failed piece downloads and error detection.
    \end{itemize}

\end{enumerate}

\section{Tracker-Specific Functions}

\begin{enumerate}[label=\textbf{\arabic*.}, leftmargin=30pt, itemsep=10pt]

    \item \textbf{Tracker Registration}  
    \textit{Description:} Registers new peers with the tracker to enable them to participate in the file-sharing network.  
    \textit{Communication Protocols:}
    \begin{itemize}
        \item \textbf{BitTorrent Tracker Protocol (HTTP):} Used to register peers with a centralized tracker.
    \end{itemize}

    \item \textbf{Tracker Announce}  
    \textit{Description:} Updates the tracker with a peer’s status, such as upload/download progress, which files the client is seeding and connection status.  
    \textit{Communication Protocols:}
    \begin{itemize}
        \item \textbf{BitTorrent Tracker Protocol (HTTP):} Communicates the peer’s status to the tracker.
    \end{itemize}

    \item \textbf{Tracker Peer List Retrieval}  
    \textit{Description:} Allows peers to retrieve a list of other peers sharing the same file from the tracker.  
    \textit{Communication Protocols:}
    \begin{itemize}
        \item \textbf{BitTorrent Tracker Protocol (HTTP):} Provides a list of active peers from the tracker.
    \end{itemize}

\end{enumerate}


\end{document}
