\documentclass{article}

% Package for setting up custom headers and footers
\usepackage{fancyhdr}
\pagestyle{fancy}
\fancyhf{}
\fancyhead[L]{Dummy LaTeX Document}
\fancyfoot[C]{\thepage}

% Title and author information
\title{Dummy LaTeX Document}
\author{John Doe}
\date{\today}

\begin{document}

% Title page
\maketitle

% Abstract section
\begin{abstract}
This is a sample abstract. Here, you provide a brief summary of the document's purpose, main points, and conclusions. It is typically short, between 150-250 words.
\end{abstract}

% Introduction
\section{Introduction}
This is the introduction section. In this section, you introduce the topic and provide some background information.

\subsection{Motivation}
In this subsection, you explain the motivation for the document. Why is this topic important? What are the goals?

\subsection{Objectives}
In this subsection, you outline the specific objectives that this document aims to achieve.

% Main Content
\section{Main Content}
Here is where you add the main content of the document. You can add more sections and subsections as needed.

\subsection{Example Subsection}
This is an example of a subsection. You can include equations, figures, and tables as necessary.

\begin{equation}
E = mc^2
\end{equation}

\begin{table}[h!]
    \centering
    \begin{tabular}{|c|c|}
        \hline
        Column 1 & Column 2 \\
        \hline
        Data 1 & Data 2 \\
        Data 3 & Data 4 \\
        \hline
    \end{tabular}
    \caption{Example Table}
    \label{tab:example}
\end{table}

% Conclusion
\section{Conclusion}
This is the conclusion section. Here, you summarize the document and highlight key takeaways.

% Bibliography
\begin{thebibliography}{9}
    \bibitem{sample} A. Author, \textit{Title of the Book}, Publisher, Year.
\end{thebibliography}

\end{document}
